\documentclass[a4paper]{article}
\usepackage[margin=5mm,top=20mm]{geometry}

\usepackage[utf8]{inputenc}

\usepackage{microtype}

\usepackage{graphicx}

\usepackage{color}

\usepackage{tikz}

\usepackage{pifont}

\usepackage{fourier-orns}
\definecolor{teal}{RGB}{51,179,198}
\definecolor{tan}{RGB}{219,143,57}

\definecolor{pink}{RGB}{255,0,212}
\definecolor{orange}{RGB}{255,128,0}
\definecolor{red}{RGB}{255,51,51}
\definecolor{darkred}{RGB}{204,0,0}
\definecolor{green}{RGB}{0,128,0}
\definecolor{capri}{RGB}{0,170,255}
\definecolor{gray}{RGB}{175,175,175}
\definecolor{lightgray}{RGB}{225,225,225}

\definecolor{softgreen}{RGB}{150,212,175}

% Sizes
\pgfmathsetmacro{\cardWidth}{6.3}
\pgfmathsetmacro{\cardHeight}{8.8}

\pgfmathsetmacro{\cardMargin}{0.25}
\pgfmathsetmacro{\valueCircleRadius}{0.08}
\pgfmathsetmacro{\costPrimaryBarLength}{0.6}
\pgfmathsetmacro{\costPrimaryBarHeight}{0.15}
\pgfmathsetmacro{\costSecondaryBarLength}{0.35}
\pgfmathsetmacro{\costSecondaryBarHeight}{0.06}
\pgfmathsetmacro{\costBarDifferenceDistance}{0.06}
\pgfmathsetmacro{\costBarDistances}{0.5}

\pgfmathsetmacro{\primaryAbilityBoxHeight}{1}

\pgfmathsetmacro{\opacity}{0.4}


% Shapes
\def\shapeCard{(0,0) rectangle (\cardWidth,\cardHeight)}
\def\shapeCardInternal{(\cardMargin,\cardMargin) rectangle (\cardWidth-\cardMargin,\cardHeight-\cardMargin)}
\def\shapeCardTextBox{(1.5*\cardMargin, 1.5*\cardMargin) rectangle (\cardWidth-1.5*\cardMargin, 0.25*\cardHeight)}

\tikzset{
	cardcorners/.style={
		rounded corners=0.2cm
	}
}


% Commands
\newcommand{\cardBorder}{
	\draw[lightgray,cardcorners] \shapeCard;
	\draw[black,cardcorners] \shapeCardInternal;
	\draw[gray, fill=lightgray] \shapeCardTextBox;
}

\newcommand{\colouredCardBorder}{	
	\draw[lightgray,cardcorners, fill=softgreen] \shapeCard;
	\draw[black,cardcorners, fill=white] \shapeCardInternal;
	\draw[gray, fill=lightgray] \shapeCardTextBox;
}

\newcommand{\cardName}[1]{
	\node at (0.5*\cardWidth, \cardHeight-5.5*\cardMargin) {\scalebox{1.3}{\bf{#1}}};
}

\newcommand{\cardType}[1]{
	\node at (0.5*\cardWidth, \cardHeight-6.7*\cardMargin) {\scalebox{0.8}{#1}};
}

\newcommand{\cardText}[1]{
	\draw \shapeCardTextBox node[pos=.5, text width=5cm, align=center] {#1};
}

\newcommand{\levelIcon}{
	\node at (0.5*\cardWidth, \cardHeight-2.5*\cardMargin) {\includegraphics[width=30pt]{images/level.png}};
}

\newcommand{\cardPower}[1]{
	\node at (0.5*\cardWidth, 0.25*\cardHeight+1.5*\cardMargin) {\scalebox{1.2}{\bf{#1}}};
}

\newcommand{\image}[1]{
	\node at (0.5*\cardWidth, 0.5*\cardHeight) {\includegraphics[width=75pt]{#1}};
}



\begin{document}
\begin{center}
\pagestyle{empty}
\hyphenpenalty=10000

\noindent
\begin{tikzpicture}
	\cardBorder
	\levelIcon
	\cardName{Bison}
	\cardType{Animal}
	\image{images/bison.png}
	\cardText{Pre combat; You must wound any stack immediately in front of the Bison.}
	\cardPower{2/4/6}
\end{tikzpicture}%
\begin{tikzpicture}
	\cardBorder
	\levelIcon
	\cardName{Flamingo}
	\cardType{Animal}
	\image{images/flamingo.png}
	\cardText{Combat; If wounded, give 1/2/3 power to the two friends in front of this stack.}
	\cardPower{0/1/2}
\end{tikzpicture}%
\begin{tikzpicture}
	\cardBorder
	\levelIcon
	\cardName{Monkey}
	\cardType{Animal}
	\image{images/monkey.png}
	\cardText{Combat; Give 2/4/6 power to front friend.}
	\cardPower{0/1/2}
\end{tikzpicture}%

\noindent
\begin{tikzpicture}
	\cardBorder
	\levelIcon
	\cardName{Pig}
	\cardType{Animal}
	\image{images/pig.png}
	\cardText{When Sold; You may immediately draft 1/1/2 cards from the market.}
	\cardPower{1/2/4}
\end{tikzpicture}%
\begin{tikzpicture}
	\cardBorder
	\levelIcon
	\cardName{Scorpion}
	\cardType{Animal}
	\image{images/scorpion.png}
	\cardText{Pre combat; All animals in the same column as this scorpion are wounded.}
	\cardPower{2/4/6}
\end{tikzpicture}%
\begin{tikzpicture}
	\cardBorder
	\levelIcon
	\cardName{Carrion}
	\cardType{Animal}
	\image{images/carrion.png}
	\cardText{Combat; Add 1/2/3 power for each wounded friend.}
	\cardPower{1/2/3}
\end{tikzpicture}%

\noindent
\begin{tikzpicture}
	\cardBorder
	\levelIcon
	\cardName{Puffin}
	\cardType{Animal}
	\image{images/puffin.png}
	\cardText{Draft; Before drafting, you may draw up to 1/2/3 cards from the deck. All cards must be kept.}
	\cardPower{0/1/2}
\end{tikzpicture}%
\begin{tikzpicture}
	\cardBorder
	\levelIcon
	\cardName{Frog}
	\cardType{Animal}
	\image{images/frog.png}
	\cardText{Draft; The frog may replace an existing stack, then replace that entire stack into your tableau as if just drafted.}
	\cardPower{1/2/3}
\end{tikzpicture}%
\begin{tikzpicture}
	\cardBorder
	\levelIcon
	\cardName{Cat}
	\cardType{Animal}
	\image{images/cat.png}
	\cardText{When sold: Earn an additional 0/1/2 victory points.}
	\cardPower{0/1/2}
\end{tikzpicture}%

\noindent
\begin{tikzpicture}
	\cardBorder
	\levelIcon
	\cardName{Mosquito}
	\cardType{Animal}
	\image{images/mosquito.png}
	\cardText{Pre combat; Roll one die. All stacks of level 1/2/3 in the rolled column are wounded. On a 6 you choose the column.}
	\cardPower{0/1/2}
\end{tikzpicture}%
\begin{tikzpicture}
	\cardBorder
	\levelIcon
	\cardName{Horse}
	\cardType{Animal}
	\image{images/horse.png}
	\cardText{Combat; Get additional 1/2/3 power for each unwounded \textit{neigh}bour.}
	\cardPower{0/1/2}
\end{tikzpicture}%
\begin{tikzpicture}
	\cardBorder
	\levelIcon
	\cardName{Bee}
	\cardType{Animal}
	\image{images/bee.png}
	\cardText{Pre combat; You may wound the stack immediately behind this bee.}
	\cardPower{2/3/4}
\end{tikzpicture}%

\noindent
\begin{tikzpicture}
	\cardBorder
	\levelIcon
	\cardName{Dog}
	\cardType{Animal}
	\image{images/dog.png}
	\cardText{Combat: Gain additional 1/2/3 power for each empty space in your tableau.}
	\cardPower{0/1/2}
\end{tikzpicture}%
\begin{tikzpicture}
	\colouredCardBorder
	\levelIcon
	\cardName{Chocolate}
	\cardType{Food}
	\image{images/chocolate.png}
	\cardText{When equipped: Tuck this chocolate under the equipped stack. Chocolate count as an animal of that type.}
\end{tikzpicture}%
\begin{tikzpicture}
	\colouredCardBorder
	\cardName{Honey}
	\cardType{Food}
	\image{images/honey.png}
	\cardText{When wounded: You may immediately sell this stack. If you do, you may draft 1/1/2 cards from the market.}
\end{tikzpicture}%

\noindent
\begin{tikzpicture}
	\colouredCardBorder
	\cardName{Sushi}
	\cardType{Food}
	\image{images/sushi.png}
	\cardText{The equipped stack gain 3 power.}
\end{tikzpicture}%
\begin{tikzpicture}
	\colouredCardBorder
	\cardName{Melon}
	\cardType{Food}
	\image{images/melon.png}
	\cardText{The equipped stack gains 1 power and cannot be wounded.}
\end{tikzpicture}%

\end{center}
\end{document}
